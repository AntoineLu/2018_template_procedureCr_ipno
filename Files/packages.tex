% PACKAGES UTILISÉS
% Vous pouvez ajouter des packages si besoin
% Évitez les packages trop exotiques ou connus pour poser des problèmes/incompatibilités
% \usepackage[T1]{fontenc}%,OT1 ?
% \usepackage[utf8x]{inputenc}
\usepackage[utf8]{inputenc}
\usepackage[T1]{fontenc}
\usepackage{geometry}%marges de la page

\usepackage{configuration}
\usepackage{lmodern}
\usepackage{xstring}

%Nouveaux styles chapitres qui enlève le gros "CHAPITRE"
\usepackage{titlesec}
\titleformat{\chapter}[hang]{\bf\huge}{\thechapter}{2pc}{}

\usepackage{amsmath}
\usepackage{microtype}
\usepackage{amsfonts}
\usepackage{amssymb}
\usepackage{amsthm}
\usepackage{mathdots}

\usepackage[dvipsnames]{xcolor}

\usepackage{siunitx}%affichage des nombres à virgules et des unités, consulter la documentation en ligne
\sisetup{
    %separate-uncertainty,%cf la documentation du package
    detect-family,
    alsoload = synchem,
    locale = FR,
    scientific-notation = false,
    exponent-product = \ensuremath{{}\cdot{}},%\times,
    inter-unit-product = \ensuremath{{}\cdot{}},
    range-units = single,
    list-units = single
}

\usepackage{graphicx}

\usepackage{tikz}% dessiner en LaTeX

\usepackage{scrextend}

\usepackage{pdfpages}% pour inclure des pages entières de pdf

\usepackage{framed}% environnements encadrés

\usepackage{hyperref}% permet de cliquer sur le sommaire et les figures et liens cliquables

\usepackage[version=4]{mhchem}% affichage des formules chimiques

\usepackage[normalem]{ulem}% sous-lignage et sur-lignage \normalem Normal emphasis \ULforem pour revenir à une emphase soulignée. Problème du package ulem. (peut également s'écrire dans le texte)

\usepackage{pgfplots}% plots directement dans latex
\pgfplotsset{compat = 1.15, height = 8cm, width = 6.5in}
% \usepgfplotslibrary{external}% convertir en image
% \tikzexternalize% convertit les plots en image

\usepackage[all]{hypcap}

\usepackage{float}
\floatplacement{figure}{!h!tbp}%placement par défaut des figures !h!tbp ou H pour ici (fort)

\usepackage{caption}%nouvelles options pour les légendes
\captionsetup{
    width = .93\textwidth
}

% Définition des en-tête
\usepackage{fancyhdr}
%\pagestyle{fancy}
%\fancyhead[C]{} 
%\fancyhead[L]{}
%\fancyhead[R]{}

% Méta-données du fichier, package hyperref (consulter documentation pour plus de détails)
\hypersetup{
    pdfauthor=\redacteur,
    pdftitle=\titre,
    pdfsubject=\type,
    pdfkeywords={Institut de physique nucléaire d'Orsay},
    pdfproducer={pdfLaTeX},
    pdfcreator={pdfLaTeX with hyperref package via ShareLaTeX},
    pdfdisplaydoctitle=true
}

% Configuration du langage de programmation si l'on souhaite importer un unique language (cf. doc listings)
\usepackage{textcomp,upquote,listings}%Pour lister du code informatique, cf. doc
\lstset{
    %language=Maple,% Choisir le langage de programmation (peut-être choisi individuellement à chaque citation).
    keywordstyle=\textbf,
    numbers=left,
    numbersep=5pt,
    basicstyle=\small,
    numberstyle=\tiny,
    upquote=true,
    showstringspaces=false,
    %inputencoding=utf8,
    showspaces=false
}
%\lstset{literate={->}{{$\rightarrow$}}1 {*}{{$\cdot$}}1}% Pour remplacer des caractères dans des lignes de code

%\usepackage{fontspec}
%\setmainfont{DejaVu Serif}
%\setmonofont{DejaVu Sans Mono}

%\usepackage{cmbright}

\usepackage{layouts}

\usepackage{import}

\usepackage[xindy, toc, nopostdot, nogroupskip]{glossaries}%glossaires
\makeglossaries
\loadglsentries{glossaire}

\usepackage{filecontents}
\usepackage{csquotes}

%%% Pour la définition des archives hal
\begin{filecontents*}{eprint-hal_id.dbx}
\ProvidesFile{eprint-hal_id.dbx}[2018/09/26 HAL/TEL eprints]
\DeclareDatamodelFields[type=field,datatype=verbatim]{arxiv,hal_id}
\DeclareDatamodelEntryfields{hal_id}
\DeclareDatamodelFields[type=field,datatype=literal]{arxivclass}
\DeclareDatamodelEntryfields{arxivclass}
\end{filecontents*}

\usepackage[
    backend=biber,
    style=numeric,
    sorting=ynt,
    datamodel=eprint-hal_id,
]{biblatex}

\usepackage{babel}
\frenchbsetup{StandardLists=true}

\usepackage{hyperref}

\DefineBibliographyStrings{french}{in={dans},inseries={dans}}
\DeclareFieldFormat{hal_id}{%
  \mkbibacro{HAL}\addcolon\space
  \ifhyperref
    {\href{https://hal.archives-ouvertes.fr/#1}{\nolinkurl{#1}}}
    {\nolinkurl{#1}}}

\DeclareFieldAlias{eprint:hal_id}{hal_id}
\DeclareFieldAlias{eprint:HAL}{eprint:hal}

\renewbibmacro*{eprint}{%
  \printfield{hal_id}%
  \newunit\newblock
  \iffieldundef{eprinttype}
    {\printfield{eprint}}
    {\printfield[eprint:\strfield{eprinttype}]{eprint}}}
%%%
\addbibresource{\nombibliographie}
% \addbibresource{bibliographie_manuelle.bib}

\usepackage{fontawesome}
\usepackage{xcolor}

\usepackage{csvsimple}% Tableaux à partir de CSV

% \newif\ifdraft

\usepackage{typearea,afterpage}% Changer la taille des pages au milieu des documents

\usepackage[lofdepth,lotdepth]{subfig} % Sous figures (plus performant)

\geometry{left=2cm, right=2cm, top=2cm, bottom=2cm}
